\documentclass[12pt,a4paper]{report}
\usepackage[utf8]{inputenc}
\usepackage[swedish]{babel}
\usepackage{fullpage}
\title{Beräkningsvetenskap och Analys 1TD333,\\ miniprojekt 3}
\begin{document}
\author{Oskar Persson, Martin Johansson, Erik Englund}
\maketitle
\tableofcontents
\section{Inledning}
Denna rapport syftar till att redogöra för, samt lösa problemet med dammen vid Newton's Mill i England. Problemet var att dammen behövde renoveras, mer specifikt att dammens väggar skulle förstärkas. Syftet med projektet var att beräkna hur dessa väggar då skulle dimensioneras för att hålla emot det hydrostatiska tryck som vattnet i dammen utövar på fördämningen. Projektet delades in i tre delar:\\
\begin{enumerate}
\item Att beräkna det befintliga hydrostatiska trycket i dammen med hjälp av given formel och mätvärden (integrering)\\
\item Att beräkna hur stort det hydrostatiska trycket kommer bli efter att väggarna förstärkts och formen på dammen förändrats. Detta genom numerisk beräkning av hydrostatiskt tryck i förhållande till aktuell vattennivå (med hjälp av matlab-funktioner).\\ 
\item Att skapa 
\end{enumerate}
\section{Problemlösning}
\subsection{Del 1}

\subsection{Del 2}
\subsection{Del 3}
\section{Resultat}
\section{Diskussion}
\section{Reflektion}
\section{Referenser}
\section{Appendix}
\end{document}